%%%%%%%%%%%%%%%%%%%%%%%%%%%%%%%%%%%%%%%%%
% Weekly Report 
% LaTeX Template
% Version 1.3 (19/10/24)
% Modified by
% Enes TAŞTAN
% Erdem TUNA
% Halil TEMURTAŞ
%%%%%%%%%%%%%%%%%%%%%%%%%%%%%%%%%%%%%%%%%
%
%----------------------------------------------------------------------------------------
%	PACKAGES AND OTHER DOCUMENT CONFIGURATIONS
%----------------------------------------------------------------------------------------
\documentclass[a4paper,12pt]{article}
%-----packages------
\usepackage[a4paper, total={6.2in, 8.5in}, headheight=110pt]{geometry}
\usepackage[english]{babel}
\usepackage[utf8x]{inputenc}
\usepackage{amsmath}
\usepackage{graphicx}
\usepackage[colorinlistoftodos]{todonotes}
\usepackage{gensymb} % this could be problem
\usepackage{float}
\usepackage{fancyref}
\usepackage{subcaption}
\usepackage[toc,page]{appendix} %appendix package
\usepackage{xcolor}
\usepackage{listings}


\usepackage[export]{adjustbox}

\usepackage{xspace}
\usepackage{amssymb}
\usepackage{nicefrac}
\usepackage{gensymb}
\usepackage{fancyhdr}
\usepackage{lipsum}  % for lipsum
\usepackage[final]{pdfpages}  % pdf include
\usepackage{array} %allows more options in tables
\usepackage{pgfplots,pgf,tikz} %coding plots in latex
\usepackage{capt-of} % allows caption outside the figure environment
\usepackage[export]{adjustbox} %more options for adjusting the images
\usepackage{multicol,multirow,slashbox} % allows tables like table1
%\usepackage[hyperfootnotes=false]{hyperref} % clickable references
\usepackage{epstopdf} % useful when matlab is involved
%\usepackage{placeins} % prevents the text after figure to go above figure with \FloatBarrier 
%\usepackage{listingsutf8,mcode} %import .m or any other code file mcode is for matlab highlighting

%-----end of packages
\input{configuration.tex}


\pagestyle{fancy}
\setlength\headheight{150pt}
\setlength{\footskip}{2.5cm}
%\fancyhead[LO,LE]{Duayenler Ltd. Şti.}
%\fancyhead[RO,RE]{October 19, 2018}
\fancyhead[LO,LE]{\textbf{Project:} IOT Distributed Temperature\\ Measurement and Control (Control Lab) }
\fancyhead[RO,RE]{Erdem Tuna\\2167419}

%\fancyhead[RO]{Sarper Sertel (05435156039),\\Enes Taştan (05436834336), Erdem Tuna (05352563320),\\Halil Temurtaş (05316322194), İlker Sağlık (05417229573)}
%\rfoot{\includegraphics[width=2.2cm]{../../../Documents/logos/logo2-page-with-stroke}}

\begin{document}
	



%\begin{figure}[ht]
%	\begin{minipage}[]{0.65\textwidth}
%		\begin{flushleft}
%			%\vspace*{-.7cm}
%\Large\textbf{A Search on Transducers for Ambient Noise Measurements}
%	\end{flushleft}		
%\end{minipage}
%\begin{minipage}[]{0.35\textwidth}
%	\begin{flushright}
%		\includegraphics[width=5cm,height=\textheight,keepaspectratio]{Breeze-Logo-Header}
%		%\caption{test}
%		\label{GIS_Mapping_Software}
%	\end{flushright}
%\end{minipage}%
%\end{figure}

\section{Introduction}
This report aims to give instructions on the STAR project "IOT Distributed Temperature Control". The report will outline the main components, developments and concepts that are used until now. To wrap the procedure up, DHT11 sensors are used to measure the temperature of the ambient. Temperature data are sent with ESP8266-01 Wi-Fi modules. The data are aggregated on a ESP8266 (over a MQTT server) that is deployed on an Arduino.
\section{DHT11 Temperature Sensors}
These sensors are famous for prototyping purposes. They are cheap in compared to similar sensors and can output digital data. However, these sensors \cite{ref-dht11} are accurate up to 2 \degree C. Sensors can measure also humidity up to 5\% accuracy. I bought several of such sensors. The current problem is that their measurements differ by $\pm$1 degrees from each other, even in the same location.

Note that, for the controller to operate correctly, sensor must send data every 80-100 ms.

\section{ESP8266-01 Wi-Fi Module}
This Wi-Fi module is also widely used in prototyping phases in the community. I bought 3 of them. The modules can connect to Wi-Fi networks that support WPA or WPA2 protocols. The problem with these modules was to connect "meturoam" network. 

The school network supports WPA2-Enterprise protocol. There was an ongoing discussion on GitHub \cite{ref-github-esp8266} whether this module can connect to WPA2-Enterprise networks. It seems too few people could manage it. Instead of using "meturoam" or "eduroam" (which uses also WPA2-Enterprise protocol), I used a local laptop in Control Laboratory. This laptop runs Windows10 OS and is connected to "meturoam". Windows10 supports broadcasting hotspot Wi-Fi networks. By using this feature, I could make ESP8266-01 module to connect internet in Control Laboratory. The successful connection is shown in \textit{Figure~\ref{fig:laptop-esp-wifi}}.

Since ESP modules are connected to the intenet, data transfer can be done. For this purpose, I chose light-weight messaging protocol MQTT \cite{ref-mqtt}. Shortly, MQTT uses publish-subscribe system via a broker(a server indeed). In this network, there are publishers that send some message including a topic, subscribers that look for messages for specific topic and a broker that manages publish-subscribe network. It is good to note that there are couple public MQTT brokers that can be used all around the world. But such brokers may introduce an average delay of 1.5 minutes in data transfer. The libraries to support this protocol are already developed by community.

\begin{figure}[t]
	\center
	\setlength{\unitlength}{\textwidth} 
	\includegraphics[width=0.4\unitlength]{images/laptop-esp-wifi}
	\caption{\label{fig:laptop-esp-wifi}Connection of ESP Modules to Laptop Hotspot.}
\end{figure}

\section{Deploying DHT11 on ESP8266}
DHT11 must be connected to the GPIO pin of the ESP module so that the temperature can be sent over Wi-Fi. This process is realized on a breadboard and explained in my GitHub repository \cite{ref-starGithub} that is dedicated for this STAR Project.
\begin{figure}[h]
	\center
	\setlength{\unitlength}{\textwidth} 
	\includegraphics[width=0.4\unitlength]{images/esp-modules}
	\caption{\label{fig:esp-modules}Deployment of DHT11 and ESP8266.}
\end{figure}
\section{Data Transfer Over Wi-Fi}
To complete the network, a private MQTT server must be open. This is easily done by employing a Google Cloud computer. "MQTT Broker" is installed on this virtual computer and the external IP traffic is enabled. The IP of this computer is used to send message from sensor to the virtual computer (broker).

On this computer, only one software runs: An MQTT subscriber and an MQTT publisher. Subscriber receives messages from sensors and publishes it at the same time.

The ESP8266 module that is deployed on the controller Arduino is also subscribed to virtual computer to receive the published messages from virtual computer.

This way, the controller can receive temperatures in a systematic way.


\section{Controller Code}

PID controller is written but MQTT is not integrated. It can be done easily by combining sensor code (MQTT part) into controller. The code is in my GitHub repository \cite{ref-starGithub} that is dedicated for this STAR Project.


\newpage
\begin{thebibliography}{}
	\addcontentsline{toc}{section}{References}
	
	%\bibitem{ex1} "Article Title," Website Title.
	% [Online].
	%  Available: website.
	% [Accessed: 24-Sep-2017].
	
	%\bibitem{mic-ref} Agarwal, Anant. Foundations of Analog and Digital Electronic %Circuits.Department of Electrical Engineering and Computer Science, Massachusetts %Institute of Technology, 2005, p. 43
	 \bibitem{ref-dht11} [Online].
	 Available: https://www.adafruit.com/product/386.
	 [Accessed: 30-Dec-2018].
	 
	 \bibitem{ref-github-esp8266} [Online].
	 Available: https://github.com/esp8266/Arduino/issues/1032.
	 [Accessed: 30-Dec-2018].
	 
	 \bibitem{ref-mqtt} [Online].
	Available: http://mqtt.org/.
	[Accessed: 30-Dec-2018].
	 
	 \bibitem{ref-starGithub} [Online].
	Available: https://github.com/erdemtuna/METU-EE-STAR-Project.
	[Accessed: 30-Dec-2018].
	
	 \bibitem{ref-starGithub} [Online].
Available: http://www.mqtt-dashboard.com/http://www.mqtt-dashboard.com/.
[Accessed: 30-Dec-2018].


		
\end{thebibliography}

\end{document}

%----samples------
%\begin{itemize}
%\item Item
%\item Item
%\end{itemize}

%\begin{figure}[H]
%\center
%\setlength{\unitlength}{\textwidth} 
%\includegraphics[width=0.7\unitlength]{images/logo1}
%\caption{\label{fig:logo}Logo }
%\end{figure}

%\begin{figure}[H]
%	\setlength{\unitlength}{\textwidth} 
%	\centering
%	\begin{subfigure}{.5\textwidth}
%  		\centering
%  		\includegraphics[width=0.48\unitlength]{images/logo1}
%  		\caption{\label{fig:logo1}Logo1 }
%	\end{subfigure}%
%	\begin{subfigure}{.5\textwidth}
%  		\centering
%		\includegraphics[width=0.48\unitlength]{images/logo2}
%  		\caption{\label{fig:logo2}Logo2}
%	\end{subfigure}
%\caption{\label{fig:calisandegree} Small Logos   }
%\end{figure}
	
%\begin{table}[H]
%  \centering
% 
%    \begin{tabular}{c|c|c}
%       $$A$$ & $$B$$ & $$C$$ \\ \hline
%       1 & 2 & 3  \\ \hline
%       2 & 3 & 4  \\ \hline
%       3 & 4 & 5  \\ \hline
%       4 & 5 & 6  
%      
%  \end{tabular}
%  \caption{table}
%  \label{tab:table}
%\end{table}
	
%\begin{table}[H]
%  \centering
% 
%    \begin{tabular}{c|c|c}
%       \backslashbox{$A$}{$a$} & $$\specialcell{ Average deviation \\ after subtracting out the  \\ frequency error }$$ & $$C$$ \\ \hline
%       \multirow{2}{*}{1} & 2 & 3  \\ \cline{2-3}
%        & 3 & 4  \\ \hline
%       3 & \multicolumn{2}{c}{4}  \\ \hline
%       4 & 5 & 6  
%      
%  \end{tabular}
%  \caption{table}
%  \label{tab:table}
%\end{table}
%-----end of samples-----